\documentclass{article}
\usepackage{lmodern}
\usepackage[T1]{fontenc}
\usepackage[utf8]{inputenc}
\usepackage[left=2.5cm,right=3cm,top=2.5cm,bottom=2.5cm]{geometry}
\usepackage{graphicx}
\usepackage{subfigure}
\usepackage{amsmath}
\usepackage{amsfonts}
\usepackage{multicol}
\usepackage{euscript}
\usepackage{enumerate}
\usepackage{wrapfig}
\usepackage[usenames]{color}

\nofiles

\title{Apuntes de \LaTeX}
\author{Daniel Michel Pino González\\
\\
Facultad de CC. Físicas\\					
Universidad Complutense de Madrid}				

\newtheorem{teo}{Teorema}[section]
\newtheorem{teo*}{Teorema}
\newtheorem{lem}{Lema}[section]						
\newtheorem{deff}{Definición}[section]		
\newtheorem{dem}{Demostración}						
\newtheorem{prop}{Propiedad}[section]			
\newtheorem{props}{Propiedades}[section]	
\newtheorem{cor}{Corolario}	
\newtheorem{parr}{}

\renewcommand{\labelitemi}{$\circ$}
\renewcommand{\labelitemii}{$\cdot$}

\setlength{\parskip}{2mm}
\setlength{\columnsep}{7mm}

\begin{document}
\maketitle
\thispagestyle{empty}
\clearpage

\section{Introducción al uso de \LaTeX}

\LaTeX es un sistema de composición de textos orientado a la creación de documentos escritos con una alta calidad tipográfica. Su uso es especialmente interesante para la generación de artículos y libros científicos por su facilidad a la hora de incluir notación matemática en el texto. Por ello, es ampliamente utilizado en ramas como la física, matemáticas, informática, biología o incluso música.

\subsection{Entornos de edición}

\subsubsection{Editores en línea: Overleaf}

\subsubsection{Editores off-line: Texmaker}

Texmaker es un editor de distribución gratuita bajo licencia GPL (GNU General Public License) que integra muchas herramientas necesarias para el desarrollo de documentos con \LaTeX.

Incluye soporte Unicode, corrección ortográfica, autocompletado de comandos, plegado de código y un visor incorportado en pdf con modo de visualización continua.

Para instalarlo, según nuestro sistema operativo:

\begin{itemize}

\item \textbf{Windows:} para su última versión, Texmaker 5.0.4 posee soporte para Windows 7, 8 y 10 en sistemas de 64 bits, para los cuales también existe una versión portable. Asimismo, existen versiones anteriores, como Texmaker 4.5, que soporta, además, Windows XP y Vista en arquitectura de 32 bits.

\item \textbf{MacOS:} se dispone de la última versión, Texmaker 5.0.4, para sistemas MacOSX (10.12 en adelante) con arquitectura de 64 bits. Para versiones anteriores, existen otras versiones, como Texmaker 4.5, con soporte para MacOSXLion de 64 bits.

\item \textbf{Linux:} Texmarker 5.0.4 da soporte a las versiones de Ubuntu 20.10, 20.04 y 19.04 con arquitectura de 64 bits, así como a Debian 10 y 9.

\bigskip

\end{itemize}

La instalación para cualquier sistema operativo es inmediata descargando su respectivo instalador en la página web oficial: https://www.xm1math.net/texmaker/download.html

\subsubsection{Compilación, errores y visualización}

\section{Preámbulo de documentos}

\subsection{Tipos de documentos}

\subsection{Numeración y formateado de páginas}

\subsection{Inclusión de paquetes}

\section{Conceptos básicos de escritura}

\subsection{Márgenes y notas}

\subsubsection{Cambio del tamaño de los márgenes}

\subsubsection{Notas a pie de página}

\subsubsection{Notas marginales}

\subsection{Índices}

\subsection{Manejo de espacios, saltos de línea y párrafos}

\subsubsection{Espacios verticales}

\subsubsection{Párrafos e indexación}

\section{Listas y enumeraciones}

\subsection{Listas enumeradas}

\subsection{Listas no enumeradas}

\section{Tablas}

\section{Figuras y elementos flotantes}
\clearpage
\section{Símbolos y fórmulas matemáticas}

\subsection{Letras griegas}

\subsection{Fórmulas y ecuaciones}

\subsection{Símbolos y operadores matemáticos}

\subsection{Vectores, matrices y determinantes}

\subsection{Funciones especiales}

\subsection{Fuentes en modo matemático}

\subsection{Teoremas}

\section{Uso de colores}

\section{Latex y artículos científicos}

\subsection{Tipo de documento y formato de texto}

\subsection{Abstracts}

\end{document}
